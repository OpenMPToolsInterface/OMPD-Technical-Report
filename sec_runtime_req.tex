\section{OpenMP Runtime Requirements}
\label{openmp-runtime-requirements:sec}

Most of the debugger's OpenMP-related activities on a target will
be performed through the OMPD interface.
However, supporting OMPD introduces some requirements of the OpenMP runtime.
Some of these have been discussed earlier.
Here we summarize these requirements and collect them together
for easy reference.

\begin{enumerate}
\item
  The OpenMP must define
  \refdef{\texttt{ompd\_dll\_locations}}{dll-locations:def};
\item
  The OpenMP must define
  \refdef{\texttt{ompd\_dll\_locations\_valid}}{dll-locations-valid:def}
  and ensure that control flows through it once
  \texttt{ompd\_dll\_locations} is ready to be read by the debugger;
\item
  In order to support debugging, the OpenMP may need to collect and
  maintain information about a target's execution that, perhaps for
  performance reasons, it would not otherwise not do.
  The OpenMP runtime must support the following mechanisms for
  indicating that it should collect whatever information is
  necessary to support OMPD:
  \begin{enumerate}
  \item
    the environment variable \texttt{OMP\_DEBUG} is set to \texttt{on};
  \item
    the \emph{target} calls
    \refdef{\texttt{omp\_debug\_enable~()}}{ompd-enable:def}
\begin{notes}
ilaguna: should OMPD support any of the previous mechanisms or both of 
them? From the text it's not clear.
\end{notes}
  \end{enumerate}
\item
  The OpenMP must define the following code symbols, and execute through
  them at the times described in Section~\ref{breakpoint-locations:sec}:
  \begin{description}
  \item [\texttt{ompd\_bp\_parallel\_begin}]
  \item [\texttt{ompd\_bp\_parallel\_end}]
  \item [\texttt{ompd\_bp\_task\_begin}]
  \item [\texttt{ompd\_bp\_task\_end}]
  \end{description}
\item
  Any OMPD-related symbols needed by the debugger must have \texttt{C} linkage.
\end{enumerate}

%% \begin{comment}
%% A debugger can control the level at which OpenMP runtime support for tools 
%%is activated by invoking


%% \begin{quote}
%% \begin{lstlisting}
%% int ompd_enable(
%%   ompd_enable_setting_t setting
%% );
%% \end{lstlisting}
%% \end{quote}

%% \noindent 
%% To disable all OMPT support for tools, a debugger calls 
%%\verb|ompd_enable(false)|.
%% To enable support for tools, 
%% a debugger  calls \verb|ompd_enable(true)|.
%% With this setting specified, an OpenMP runtime will maintain runtime state 
%%(as described in~\ref{sec:states}) and support all OMPT tool-facing inquiry 
%%functions (as described in Section~\ref{sec:inquiry}). 

%% When \verb|ompd_enable| is called, its effect is not necessarily 
%%instantaneous. A call to enable or disable tool support will take effect at a 
%%clean point.
%% % (as described in Section~\ref{sec:init}).

%% Upon a call to \verb|ompd_enable(true)|, if has not already been enabled, an 
%%OpenMP runtime may invoke a tool's \verb|ompt_initialize| callback at the 
%%next 
%%clean point.
%% Upon a call to \verb|ompd_enable| with  \verb|false| as an argument,  if a  
%%tool has already been initialized and the tool
%% has registered a callback for \verb|ompt_event_runtime_shutdown|, the 
%%shutdown callback may occur no earlier than the next clean point.

%% If a tool is already enabled before a call to \verb|ompd_enable(true)|, a 
%%call to \verb|ompt_enable_complete| occurs before the call to 
%%\verb|ompd_enable|  returns. 
%% If no tool is present or it has already been disabled, the call to 
%%\verb|ompt_enable_complete| occurs before the call to \verb|ompd_enable|  
%%returns.
%% A debugger can set a breakpoint in  \verb|ompt_enable_complete|  to observe 
%%when a tool has been enabled or disabled.
%% \end{comment}
\clearpage
