\section{Breakpoint Locations for Managing Parallel Regions and Tasks}
\label{breakpoint-locations:sec}

Neither a debugger nor an OpenMP runtime system know what
application code a program will launch as parallel regions
or tasks until the program invokes the runtime system and
provides a code address as an argument.
To help a debugger control the execution of an OpenMP program
launching parallel regions or tasks,
%%%% OMPD provides a routine that the debugger can invoke to 
%%%% determine where to place breakpoints. 
the OpenMP runtime must define a number of routines in which
the debugger may plant breakpoints to receive notification of
particular events.
The runtime is expected to call these routines when these events occur
\emph{and} data collection for OMPD is
enabled (see~\S\ref{runtime-interface:sec}).

\paragraph{Advice to implementors} The debugger needs to be able to detect 
the beginning of OpenMP runtime code.
Especially inline generated runtime code should be built without source line
information.

\begin{notes}
Ariel: What does this last sentence mean?

John: I think the intention here was to reflect that if the OpenMP
is built with line number information then a ``step into'' operation
in the debugger might step into the OpenMP runtime function
instead of ``step over'' the function.
Like with other runtime library functions, ``step into'' should act like
``step over'' for the OpenMP runtime.
In essence, we need a way to let the debugger know that the OpenMP runtime
is not part of the user's source code, and one way of doing that is to
not generate line number information for the OpenMP runtime code.
However, I'm not sure that's the best way of doing it.

Ariel:
What's the use case?
If we've hit the enter breakpoint we can find out what user code
is going to be executed by getting the function for the region.
The debugger can plant a breakpoint there and let the target run.

Or is the case that the user is stepping through his code and
steps into a function call that is part of the OpenMP runtime,
and we want to know that to zoom past that to the user code?
I.e., the problem is knowing what code is OpenMP code?
If the user continues stepping far enough the frame information for the thread
should indicate whether the routine is OpenMP code.

Is the stack exit/reentry information set up for all entries to OpenMP,
or only for those entries that result in executing user code?
E.g., if the user's code call \texttt{omp\_get\_thread\_num},
is the stack exit/reentry information set up?  Or is it only for
things like handle a parallel region construct?

So what OpenMP code are we wanting to identify?

Another thought:
if the user is stepping by source line, then if the OpenMP code is inlined,
where would we expect the debugger to advance to?
Is this is what Joachim is getting at by suggesting that there be
no line numbers for the generated code?
If the inlined code includes a call, can we detect
that the destination of the call is OpenMP?
Well, we may be able to answer that is the branch is to what we
know is the OpenMP runtime library.

Bottom line: what we want to do about this `Advice to implementors'?

\end{notes}

\subsection{Parallel Regions}

The OpenMP runtime must call \texttt{ompd\_bp\_parallel\_begin}
when a new parallel region is launched.
The call should  occur after a task encounters a parallel construct,
but before any implicit task starts to execute the parallel
region's work.
The type signature for \texttt{ompd\_bp\_parallel\_begin} is:
\begin{quote}
\begin{lstlisting}
void ompd_bp_parallel_begin ( void );
\end{lstlisting}
\end{quote}
\labeldef{bp-parallel-begin:def}

When the debugger gains control when the breakpoint is triggered,
the debugger can map the the operating system thread to an OpenMP
thread handle using
\refdef{\texttt{ompd\_get\_thread\_handle}}{get-thread-handle:def}.
At this point the handle returned by
\refdef{\texttt{ompd\_get\_top\_parallel\_region}}{get-top-parallel-region:def}
is that of the new parallel region.
The debugger can find the entry point of the user code that
the new parallel region will execute by passing the parallel handle region
to \refdef{\texttt{ompd\_get\_parallel\_function}}{get-parallel-function:def}.
The actual number of threads, rather than the requested number of threads,
in the team is returned by
\refdef{\texttt{ompd\_get\_num\_threads}}{get-num-threads:def}.
The task handle returned by
\refdef{\texttt{ompd\_get\_top\_task\_region}}{get-top-task-region:def}
will be that of the task encountering the parallel construct.
The `reenter runtime' address in the information returned by
\refdef{\texttt{ompd\_get\_task\_frame}}{get-task-frame:def}
will be that of the stack frame where the thread entered the OpenMP
runtime to handle the parallel construct.
The `exit runtime' address will be for the stack frame where the thread
left the OpenMP runtime to execute the user code that encountered
the parallel construct.

When a parallel region finishes, the OpenMP runtime will call
the \texttt{ompd\_bp\_parallel\_end} routine:
\begin{quote}
\begin{lstlisting}
void ompd_bp_parallel_end ( void );
\end{lstlisting}
\end{quote}
\labeldef{bp-parallel-end:def}
At this point the debugger can map the operating system thread that
hit the breakpoint to an OpenMP thread handle using
\refdef{\texttt{ompd\_get\_thread\_handle}}{get-thread-handle:def}.
\refdef{\texttt{ompd\_get\_top\_parallel\_region}}{get-top-parallel-region:def}
returns the handle of the terminating parallel region.
\refdef{\texttt{ompd\_get\_top\_task\_region}}{get-top-task-region:def}
returns the handle of the task that encountered the
parallel construct that initiated the parallel region just
terminating.
The `reenter runtime' address in the frame information returned by
\refdef{\texttt{ompd\_get\_task\_frame}}{get-task-frame:def}
will be that for the stack frame in which the thread entered the
OpenMP runtime to start the parallel construct just terminating.
The `exit runtime' address will refer to the stack frame where the
thread left the OpenMP runtime to execute the user code that
invoked the parallel construct just terminating.

Both the begin and end events are raised once per region,
and not once for each thread per region.


%% \begin{comment}
%% The OpenMP runtime
%% \texttt{void ompd\_bp\_parallel\_pre\_execute ( void );}
%%   The OpenMP runtime shall call this routine after a task encounters
%%   a parallel construct but before any implicit task starts to execute
%%   the parallel region's work.
%%   \footnote{What information do we expect the debugger to be able to get
%%   when this event is triggered?
%%   This event appears to correspond to the OMPT
%%   \texttt{ompt\_event\_parallel\_begin} event.
%%   In the case of OMPT, the registered callback for this event is supplied
%%   with extra information by the OpenMP runtime.
%%   This is the prototype of the callback:
%%   \begin{quote}
%%   \texttt{typedef (*ompt\_new\_parallel\_callback\_t ( /* for new parallel 
%%*/\\
%%     \hspace*{16ex}ompt\_task\_id\_t   parent\_task\_id,  /* ID of parent 
%%task */\\
%%     \hspace*{16ex}ompt\_frame\_t  *parent\_task\_frame, /* frame data if 
%%parent task */\\
%%     \hspace*{16ex}ompt\_parallel\_id\_t parallel\_id, /* ID of parallel 
%%region */\\
%%     \hspace*{16ex}uint32\_t requested\_team\_size, /* requested number of 
%%threads */\\
%%     \hspace*{16ex}void *parallel\_function   /* pointer to outlined function 
%%*/\\
%%     );
%%   }
%%   \end{quote}
%%   See \S3.1 and \S{}A.3 of the OMPT TR2.
%%   How do we expect the debugger to get this information?
%%   Do we want to make the prototype of this BP function match that
%%   of \texttt{ompt\_new\_parallel\_callback\_t}?}
%%   \marginpar{see footnote \thefootnote}
%% \begin{quote}
%% \begin{lstlisting}
%% void ompd_bp_parallel_pre_execute ( void );
%% void ompd_bp_parallel_post_execute ( void );
%% void ompd_bp_task_pre_execute ( void );
%% void ompd_bp_task_post_execute ( void );
%% \end{lstlisting}
%% \end{quote}
%% \end{comment}

%% \begin{comment}
%% The \verb|ompd_get_breakpoints| routine will fill in an  
%%\verb|ompd_breakpoints_t| structure with pointers to code locations where the 
%%debugger can place breakpoints to intercept execution just before the OpenMP 
%%runtime launches a task or parallel region, and just after execution of a 
%%parallel region or task completes. 
%% \begin{quote}
%% \begin{lstlisting}
%% typedef struct ompd_breakpoints_s {
%%   ompd_taddr_t parallel_pre_execute;
%%   ompd_taddr_t parallel_post_execute;
%%   ompd_taddr_t task_pre_execute;
%%   ompd_taddr_t task_post_execute;
%% } ompd_breakpoints_t;

%% ompd_rc_t ompd_get_breakpoints(
%%   ompd_context_t *context,  /* debugger handle for the target */
%%   ompd_breakpoints_t *bkpt_locations
%% );
%% \end{lstlisting}
%% \end{quote}
%% \end{comment}

%% \begin{comment}
%% When the debugger gains control as the 
%% \verb|ompd_bp_parallel_pre_execute| code location breakpoint  triggers,
%% the debugger can determine what user code the parallel region will
%% execute by mapping the operating system thread that triggered the
%% breakpoint to an OpenMP thread handle using \verb|ompd_get_thread_handle|.
%% It can then map the thread handle to a parallel region handle using
%% \verb|ompd_get_top_parallel_region|, and then using
%% \verb|ompd_get_parallel_function| to determine the entry point
%% for the user code that the parallel region will execute. 

%% Similarly, when the debugger gains control as a breakpoint
%% at the  \verb|ompd_task_pre_execute| code location  triggers, 
%% the debugger can determine what user task code will execute
%% by mapping a native operating system thread to an OpenMP thread handle 
%% using \verb|ompd_get_thread_handle|.
%% As before, the debugger can then map the thread handle to a parallel
%% region handle using \verb|ompd_get_top_task_region|, and then using
%% \verb|ompd_get_task_function| to determine the entry point for
%% the user tasking code.

%% Each of these breakpoints is triggered only once per parallel region,
%% not once per thread in a parallel region.
%% The \verb|ompd_bp_task_pre_execute| and
%% \verb|ompd_bp_task_post_execute| breakpoints may be triggered
%% in different threads if a task executes on a different thread
%% then where it was launched.
%% \end{comment}

\subsection{Task Regions}

When starting a new task region, the OpenMP runtime system
calls \texttt{ompd\_bp\_task\_begin}:
\begin{quote}
\begin{lstlisting}
void ompd_bp_task_begin ( void );
\end{lstlisting}
\end{quote}
\labeldef{bp-task-begin:def}
The OpenMP runtime system will call this routine after the task
construct is encountered, but before the new explicit task starts.
When the breakpoint is triggered the debugger can map the operating
thread to an OpenMP handle using
\refdef{\texttt{ompd\_get\_thread\_handle}}{get-thread-handle:def}.
\refdef{\texttt{ompd\_get\_top\_task\_region}}{get-top-task-region:def}
returns the handle of the new task region.
The entry point of the user code to be executed by the new task
from returned from
\refdef{\texttt{ompd\_get\_task\_function}}{get-task-function:def}.

When a task region completes, the OpenMP runtime system
calls the \texttt{ompd\_bp\_task\_end} function:
\begin{quote}
\begin{lstlisting}
void ompd_bp_task_end ( void );
\end{lstlisting}
\end{quote}
\labeldef{bp-task:end}
As above, when the breakpoint is hit the debugger can use
\refdef{\texttt{ompd\_get\_thread\_handle}}{get-thread-handle:def}
to map the triggering operating system thread to the corresponding
OpenMP thread handle.
At this point
\refdef{\texttt{ompd\_get\_top\_task\_region}}{get-top-task-region:def}
returns the handle for the terminating task.