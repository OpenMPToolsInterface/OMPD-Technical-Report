\section{Memory Management}
\label{memory-management:sec}

The OMPD DLL must not access the heap manager directly. Instead if it needs 
heap memory it should use the memory allocation and deallocation callback 
functions provided by the debugger to obtain and release heap memory.
This will ensure that the DLL does not interfere with any custom memory 
management scheme the debugger may use.

If the OMPD DLL is implemented in \texttt{C++}, memory management operators 
like \texttt{new} and \texttt{delete} in all their variants, \emph{must all} be 
overloaded and implemented in terms of the callbacks provided by the debugger.

In some cases the OMPD DLL will need to allocate memory to
return results to the debugger. This memory will then be `owned' by the 
debugger, which will be responsible for releasing it. It is therefore vital 
that the OMPD DLL and the debugger use the same memory manager.

Handles are created by the OMPD implementation. These are opaque to the 
debugger, and depending on the specific implementation of OMPD may have complex 
internal structure.
The debugger cannot know whether the handle pointers 
returned by the API correspond to discrete heap allocations. Consequently, the 
debugger must not simply deallocate a handle by passing an address it receives 
from the OMPD DLL to its own memory manager. Instead, the API includes 
functions that the debugger must use when it no longer needs a handle.

Contexts are created by the debugger and passed to the OMPD implementation.
The OMPD DLL does not need to release contexts; instead this will be done by 
the debugger after it releases any handles that may be referencing the contexts.