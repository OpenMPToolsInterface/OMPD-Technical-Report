\section{Initialization and Finalization}

As described in the following sections, the OMPD DLL must be
initialized exactly once after it is leaded, and finalized exactly once
before it is unloaded. Per target process or core file initialization and
finalization are also required.

\subsection{Per DLL Initialization}

The debugger starts the initialization by calling
\texttt{ompd\_initialize}, which is defined by the OMPD DLL implementation.
Typically this will happen after the debugger has loaded the
OMPD DLL.
Once loaded, the debugger can determine the version of the OMPD
API supported by the DLL by calling the following function in the DLL:
\begin{quote}
\begin{lstlisting}
ompd_rc_t ompd_get_version ( int *version );
\end{lstlisting}
\end{quote}
On success this should return \verb|ompd_rc_ok|;
\verb|ompd_rc_bad_input| indicates that the argument is invalid.
Other errors could be reported by \verb|ompd_rc_error|.
A descriptive string describing the OMPD implementation
is returned by this function:
\begin{quote}
\begin{lstlisting}
ompd_rc_t ompd_get_version_string ( const char **string );
\end{lstlisting}
\end{quote}
The return values are the same as \verb|ompd_get_version|.
The string returned by the OMPD DLL is `owned' by the DLL,
and it must not be modified or released by the debugger.
It is guaranteed to remain valid for as long as the DLL is loaded.
\texttt{ompd\_get\_version\_string} may be called before
\texttt{ompd\_initialize} (see below).
Accordingly, the OMPD DLL must not use heap or stack memory for
the string it returns to the debugger.

The signatures of \texttt{ompd\_get\_version} and
\texttt{ompd\_get\_version\_string} are guaranteed not to change
in future version of the API.
In contrast, the type definitions and prototypes in the rest of the API
do not carry the same guarantee.
Therefore the debugger should check the version of the API of
a loaded OMPD DLL before calling any other function of the API.

The debugger must provide the OMPD library with a set of callback functions 
that enable OMPD to allocate and deallocate memory in the debugger's address 
space, to lookup the sizes of basic primitive types in the target, to lookup 
symbols in the target, as well as to read and write memory in the target. These 
callback functions are provided to the OMPD library via a table---a list of 
function pointers---of type \texttt{ompd\_callbacks\_t}.

The signature of the function is shown below:

\begin{quote}
\begin{lstlisting}
ompd_rc_t ompd_initialize ( const ompd_callbacks_t *callbacks );
\end{lstlisting}
\end{quote}
\labeldef{initialize:def}

\noindent
The type \refdef{\texttt{ompd\_callbacks\_t}}{callbacks-t:def} is defined in
Section~\ref{sec:ompd_data_types}. 
The argument is guaranteed to be valid for the duration of the call.
The OMPD library cannot assume that \texttt{callbacks} will remain valid
after the call returns back to the debugger.
\begin{notes}
ilaguna: We need to be more specific here. What does the previous sentence 
mean?	
\end{notes}

%% \begin{comment}
%% The OMPD debugger support library needs the debugger to provide a set of 
%%callback functions that enable  OMPD to manage memory in the debugger address 
%%space, look up sizes for primitive types in the target,  to look up symbols 
%%in 
%%the target, query information about structures in the target, as well as 
%%read/write memory in the target.  The OMPD library invokes the function 
%%\verb|ompd_initialize|, passing a pointer to a \verb|ompd_callbacks_t| 
%%structure that the debugger will initialize for OMPD. The signature for the 
%%function is shown below. 

%% \begin{quote}
%% \begin{lstlisting}
%% ompd_rc_t ompd_initialize (
%%   ompd_callbacks_t *data
%% );
%% \end{lstlisting}
%% \end{quote}


%% \noindent
%% The OMPD library may call \verb|ompd_initialize| in a library initialization 
%%constructor.
%% The type \verb|ompd_target_t| is defined in 
%%Appendix~\ref{sec:ompd_data_types}. 
%% \end{comment}

On success, \texttt{ompd\_initialize} returns \texttt{ompd\_rc\_ok}.
If the \texttt{data} argument is invalid, \texttt{ompd\_rc\_bad\_input}
should be returned.
All other errors will be reported by \texttt{ompd\_rc\_error}.

The above initialization is performed for each OMPD DLL that is loaded by
the debugger; there may more than one DLL present in the debugger
because it may be controlling a number of targets that may be using
different runtimes which require different OMPD DLLs.
This initialization must be performed exactly once before the debugger
can begin operating on a target process or core file.

\subsection{Per Target Initialization}

The debugger initializes a session working on a target process or core
file by calling:
\begin{quote}
\begin{lstlisting}
ompd_rc_t ompd_process_initialize (
  ompd_address_space_context_t  *context,                   /* IN */
  ompd_address_space_handle_t  **handle                    /* OUT */
);
\end{lstlisting}
\end{quote}
\labeldef{process-initialize:def}

%% \begin{center}
%% \sl
%% John: General comment: Do we want the contexts and handles to be
%% \texttt{const} in the API?
%% The debugger doesn't change handles, and the DLL doesn't change contexts,
%% so it seems like passing them as \texttt{const} across the interface
%% might make sense.
%% However, making the handles \texttt{const} is problematic because
%% if the DLL wants to modify what's cached in the handle
%% it would have to cast away the \texttt{const} to do so.
%% Which is the lesser evil?
%% Perhaps for opaque objects (like handles and contexts) \texttt{const}
%% is not needed because one side of the API does not have the definition
%% for the other side.
%% For now, I've left the API calls unchanged until we decide which way
%% is best.

%% Ariel: With respect to the comment that the DLL doesn't change contexts,
%% that may be so, but the DLL does pass contexts to the callbacks,
%% which would therefore have to have matching prototypes.
%% Given that for the opaque objects the defining side is going to have
%% to cast to the concrete definition, the \texttt{const} may have some value
%% in indicating to the user of the API what can and cannot change.
%% Having the \texttt{const} means that any variables in the debugger
%% used to hold on to handles will need to be decorated accordingly.
%% As John asks, which is the lesser evil?
%% \end{center}

The \verb|context| argument is the pointer to the debugger's host
address space context object for the target process or core file.
The OMPD implementation returns a pointer to the address space handle
in \verb|*handle|, which the debugger is responsible for releasing
when it is no longer needed.
This function must be called before any OMPD operations are performed
on the target.
\verb|ompd_process_initialize| gives the OMPD DLL an opportunity
to confirm that it is capable of handling the target process or core file 
identified
by the context.
Incompatibility is signaled by a return value of \verb|ompd_rc_incompatible|.

On return, the handle is owned by the debugger, which must release it
using
\refdef{\texttt{ompd\_release\_address\_space\_handle}}{release-address-space-handle:def}.

\subsection{Per Target Finalization}

%% Joachim: This should be rephrased to releasing language.
%%          Also: should this go to the handle management section?
%% Ariel: it should go in the handle management section for consistency ...
%%        But at the moment this is the only place where address space
%%        handles are created, so the natural place for it is here ...
%%        But in the future there will be other methods that return
%%        address space handles, so the release method should be described
%%        in the handles section.
%%        So what I've done is put text here and the definition in the
%%        handles section.

When the debugger is finished working on the target address space
for a process or core file, it calls
\refdef{\texttt{ompd\_release\_address\_space\_handle}}{release-address-space-handle:def}
to tell the OMPD implementation that it not longer needs the address space,
and to give the OMPD implementation an opportunity to release any
resources it may have related to the handle.


\subsection{Per DLL Finalization}

When the debugger is finished with the OMPD DLL it should call:
\begin{quote}
\begin{lstlisting}
ompd_rc_t ompd_finalize ( void );
\end{lstlisting}
\end{quote}
\labeldef{finalize:def}
before unloading the DLL.
This should be the last call the debugger makes to the DLL before
unloading it.
The call to \verb|ompd_finalize| gives the OMPD DLL a chance to
free up any remaining resources it may be holding.

The OMPD DLL may implement a \emph{finalizer} section.
This will execute as the DLL is unloaded, and therefore after
the debugger's call to \verb|ompd_finalize|.
The OMPD DLL is allowed to use the callbacks (provided to it earlier
by the debugger after the call to \verb|ompd_initialize|) during finalization.