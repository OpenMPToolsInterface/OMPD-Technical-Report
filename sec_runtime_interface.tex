\section{OpenMP Runtime Interface}
\label{runtime-interface:sec}

As part of the OpenMP interface, OMPD requires that the OpenMP
runtime system provides a public variable \verb|ompd_dll_locations|,
which is an \texttt{argv}-style vector of filename string pointers that
provides the pathnames(s) of any compatible OMPD plugin
implementations (if any).
\texttt{ompd\_dll\_locations} must have \texttt{C} linkage.
The debugger uses the name verbatim, and in particular, will
not apply any name mangling before performing the look up.
The pathnames may be relative or absolute. The variable declaration is as 
follows:

\begin{quote}
\begin{lstlisting}
const char **ompd_dll_locations;
\end{lstlisting}
\end{quote}
\labeldef{dll-locations:def}

\noindent
%%%% The value of \verb|ompd_dll_locations| may be NULL initially,
%%%% but must be filled in before \verb|ompd_initialize| is called.
%%%% After that,
\verb|ompd_dll_locations|  shall point to a NULL-terminated
vector of zero or more NULL-terminated pathname strings.
There are no filename conventions for pathname strings.
The last entry in the vector shall be NULL. 
The vector of string pointers must be fully initialized \emph{before}
\texttt{ompd\_dll\_locations} is set to a non-NULL value,
such that if the debugger stops execution at any point
where \texttt{ompd\_dll\_locations} is non-NULL, then the vector
of strings it points to is valid and complete.

The programming model or architecture of the debugger (and hence
that of the required OMPD) might not match that of the target
OpenMP program.
It is the responsibility of the debugger to interpret the contents
of \texttt{ompd\_dll\_locations} to find a suitable OMPD that matches
its own architectural characteristics.
On platforms that support different programming models
(\textit{e.g.}, 32- v. 64-bit), OpenMP implementers are encouraged
to provide OMPD implementations for all models, and which can handle
targets of any model.
Thus, for example, a 32-bit debugger should be able to debug a 64-bit target
by loading a 32-bit OMPD that can manage a 64-bit OpenMP runtime.

The OpenMP runtime shall notify the debugger that
\texttt{ompd\_dll\_locations} is valid by allowing execution
to pass through a location identified by the symbol
\texttt{ompd\_dll\_locations\_valid}.
This symbol must have external, \texttt{C}, linkage.

Conceptually, \texttt{ompd\_dll\_locations\_valid} has the following
signature:
\begin{quote}
\begin{lstlisting}
void ompd_dll_locations_valid ( void );
\end{lstlisting}
\end{quote}
\labeldef{dll-locations-valid:def}
However, as noted in Section~\ref{structural-overview:sec},
the event notification location does not need to be a function,
and can instead be a labeled location in the code.

\noindent
The debugger can receive notification of this event by planting
a breakpoint at this location.
\texttt{ompd\_dll\_locations\_valid} has \texttt{C} linkage,
and the debugger will not apply name mangling before searching
for this routine.
In order to support debugging, the OpenMP runtime may need to collect
and maintain information that it might otherwise not do, perhaps
for performance reasons, or because it is not otherwise needed.
The OpenMP runtime will collect whatever information is necessary
to support OMPD debugging if:
\begin{enumerate}
\item
  the environment variable \texttt{OMP\_DEBUG} is set to \texttt{on}
\item
  the target calls the
  \texttt{void omp\_debug\_enable~(~void~)}\labeldef{ompd-enable:def}
  function defined in the OpenMP runtime.
  This function may be called by the main executable, or any of the
  shared libraries the executable loads, and may be made in an
  initializer executed when a shared library is loaded
  (\textit{e.g.}, those in the \texttt{.init} section of an ELF DLL).
  It should be called before the target executes its first OMP
  construct.

  \textbf{Rationale:}
  In some cases it may not be possible to control a target's
  environment.
  \texttt{omp\_debug\_enable} allows a target itself to turn on
  data collection for OMPD.
  Allowing the function to be called from an initializer allows
  the call to be positioned in an otherwise empty DLL that the
  programmer can link with the target.
  This leaves the target code unmodified.
\end{enumerate}


%% \begin{comment}
%% If OMPT runtime state tracking has been enabled for a
%% runtime~\cite{ompt-tr2}, the OpenMP runtime system will maintain
%% information about the state of each OpenMP thread that will be
%% available to an OMPD plugin, including the following types of information:
%% \verb|ompt_state_t|, \verb|ompt_wait_id_t|, and \verb|ompt_frame_t|
%% data structures.
%% \end{comment}
