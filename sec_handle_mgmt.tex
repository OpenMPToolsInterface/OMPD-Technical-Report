\section{Handle Management}

Each OMPD call that is dependent on some context must provide this
context via a handle.
There are handles for address spaces, threads, parallel regions, and tasks.
Handles are guaranteed to be constant for the
duration of the construct they represent.
This section describes function interfaces for extracting
handle information from the OpenMP runtime system.

\subsection{Address Space Handles}

The debugger obtains an address space handle when it initializes
a session on a live process or core file by calling
\refdef{\texttt{ompd\_process\_initialize}}{process-initialize:def}.
On return from \verb|ompd_process_initialize| the address
space handle is owned by the debugger.

When the debugger is finished with the target address space handle it
should call \texttt{ompd\_release\_address\_space\_handle} to release
the handle and give the OMPD implementation the opportunity to release
any resources it may have related to the target.

\begin{quote}
\begin{lstlisting}
ompd_rc_t ompd_release_address_space_handle (
  ompd_address_space_handle_t  *handle                      /* IN */
);
\end{lstlisting}
\end{quote}
\labeldef{release-address-space-handle:def}


\subsection{Thread Handles}

\paragraph{Retrieve handles for all OpenMP threads.}
The \verb|ompd_get_threads| operation enables the debugger
to obtain pointers to handles for all OpenMP threads associated
with an address space handle.
A successful invocation of  \verb|ompd_get_threads| returns
a pointer to a vector of pointers to handles in \verb|*thread_handle_vector|
and returns the number of handle pointers in \verb|*num_handles|.
This call yields meaningful results only if all OpenMP threads in
the target process are stopped;
otherwise, the OpenMP runtime may be creating and/or destroying
threads during or after the call, rendering useless the
vector of handles returned.

\begin{quote}
\begin{lstlisting}
ompd_rc_t ompd_get_threads (
  ompd_address_space_handle_t  *handle,                     /* IN */
  ompd_thread_handle_t       ***thread_handle_vector,      /* OUT */
  int                          *num_handles                /* OUT */
);
\end{lstlisting}
\end{quote}
\labeldef{get-threads:def}

The \texttt{num\_handles} pointer argument must be valid.
The \texttt{thread\_handle\_vector} pointer argument may be NULL,
in which case the number of handles that would have been returned
had the argument not been NULL is returned in \texttt{*num\_handles}.
This allows the debugger to find out how many OpenMP threads are
running in the address space when it is not interested in the handles 
themselves.

The OMPD DLL gets the memory required for the vector of pointers to thread 
handles using the memory allocation routine in the callbacks it received during
the call to \refdef{\texttt{ompd\_initialize}}{initialize:def}. If the OMPD 
implementation needs to allocate heap memory for the thread handles, it must 
use the callbacks to acquire this memory. On return, the vector and the thread 
handles are `owned' by the debugger, and the debugger is responsible for 
releasing them when they are no longer required.

The thread handles must be released by calling 
\refdef{\texttt{ompd\_release\_thread\_handle}}{release-thread-handle:def}.
The vector was allocated by the OMPD implementation using the allocation 
routine in the callbacks it received during
initialization (see \refdef{\texttt{ompt\_initialize}}{initialize:def});
the debugger must deallocate the vector in a compatible manner.

\paragraph{Retrieve handles for OpenMP threads in a parallel region.}
The  \verb|ompd_get_thread_in_parallel| operation enables the debugger
to obtain handles for all OpenMP threads associated with a parallel region.
A successful invocation of  \verb|ompd_get_thread_in_parallel|
returns a pointer to a vector of pointers to thread handles in
\verb|*thread_handle_vector|,
and returns the number of handles in \verb|*num_handles|.
This call yields meaningful results only if all OpenMP threads
in the parallel region are stopped; otherwise, the OpenMP runtime
may be creating and/or destroying threads during or after the call,
rendering useless the vector of handles returned.

\begin{quote}
\begin{lstlisting}
ompd_rc_t ompd_get_thread_in_parallel (
  ompd_parallel_handle_t   *parallel_handle,                        /* IN */
  ompd_thread_handle_t   ***thread_handle_vector,                  /* OUT */
  int                      *num_handles                            /* OUT */
);
\end{lstlisting}
\end{quote}
\labeldef{get-thread-in-parallel:def}
The \texttt{num\_handles} pointer argument must be valid.
The \texttt{thread\_handle\_vector} pointer argument may be NULL, in which
case the number of handles that would have been returned had the argument
not been NULL is returned in \texttt{*num\_handles}.

The OMPD must obtain the memory for the vector of pointers to thread handles
using the memory allocation callback function that was passed to it
during \refdef{\texttt{ompd\_initialize}}{initialize:def}.
If the OMPD implementation needs to allocate heap memory for the
thread handles it must use the callbacks to acquire this memory.
After the call the vector and the thread handles are `owned' by the debugger,
which is responsible for releasing them.
The thread handles must be released by calling
\refdef{\texttt{ompd\_thread\_handle}}{release-thread-handle:def}.
The vector was allocated by the OMPD implementation using the allocation
routine in the callbacks; the debugger must deallocate  the vector
in a compatible manner.

\paragraph{Retrieve the handle for the OpenMP master thread in a parallel 
region.}
The  \verb|ompd_get_master_thread_in_parallel| operation
enables the debugger to obtain 
a handle for the OpenMP master thread in a parallel region. 
A successful invocation of \verb|ompd_get_master_thread_in_parallel|
returns a handle
for the thread that encountered the parallel construct. This call yields
meaningful results only if an OpenMP thread in the parallel region is
stopped; otherwise, the parallel region is not guaranteed to be alive. 
\begin{quote}
\begin{lstlisting}
ompd_rc_t ompd_get_master_thread_in_parallel (
   ompd_parallel_handle_t  *parallel_handle,                        /* IN */
   ompd_thread_handle_t   **thread_handle                          /* OUT */
); 
\end{lstlisting}
\end{quote}
\labeldef{get-master-thread-in-parallel:def}

On success \texttt{ompd\_get\_master\_thread\_in\_parallel}
returns \texttt{ompd\_rc\_ok}.
A pointer to the thread handle is returned in \texttt{*thread\_handle}.
After the call the thread handle is owned by the debugger,
which must release it when it is no longer required by calling
\refdef{\texttt{ompd\_release\_thread\_handle}}{release-thread-handle:def}.

\paragraph{Release a thread handle.}
Thread handles are opaque to the debugger, which therefore cannot
release them directly.
Instead, when the debugger is finished with a thread handle it
must pass it to the OMPD \texttt{ompd\_release\_thread\_handle}
routine for disposal.

\begin{quote}
\begin{lstlisting}
ompd_rc_t ompd_release_thread_handle (
  ompd_thread_handle_t  *thread_handle                              /* IN */
);
\end{lstlisting}
\end{quote}
\labeldef{release-thread-handle:def}

\paragraph{Compare thread handles.}
The internal structure of thread handles is opaque to the debugger.
While the debugger can easily compare pointers to thread handles,
it cannot determine whether handles of two different addresses
refer to the same underlying thread. The following function can be used to 
compare thread handles.
\begin{quote}
\begin{lstlisting}
ompd_rc_t ompd_thread_handle_compare (
  ompd_thread_handle_t   *thread_handle_1,                          /* IN */
  ompd_thread_handle_t   *thread_handle_2,                          /* IN */
  int                    *cmp_value                                /* OUT */
);
\end{lstlisting}
\end{quote}
\labeldef{thread-handle-compare:def}
On success, \texttt{ompd\_thread\_handle\_compare} returns in
\texttt{*cmp\_value} a signed integer value that indicates how
the underlying threads compare:
a value less than, equal to, or greater than 0 indicates that
the thread corresponding to \texttt{thread\_handle\_1} is,
respectively, less than, equal to, or greater than that corresponding
to \texttt{thread\_handle\_2}.

\begin{notes}
ilaguna: do we need to give intuition about what we mean by thread1 < thread2 
(or vice versa)? Will the OMPD DLL maintain a total order or a partial order of 
thread handles? If thread1 < thread2, and thread2 < thread3, is thread1 < 
thread3 or can thread1 > thread3?
\end{notes}

For OMPD implementations that always have a single, unique, underlying
thread handle for a given thread, this operation reduces to a simple
comparison of the pointers.
However, other implementations may take a different approach,
and therefore the only reliable way of determining whether two different
pointers to thread handles refer the same or distinct threads is to use
\texttt{ompd\_thread\_handle\_compare}.

Allowing thread handles to be compared allows the debugger to hold
them in ordered collections.
The means by which thread handles are ordered is implementation-defined.

\paragraph{String id.}
The \texttt{ompd\_get\_thread\_handle\_string\_id} function returns a string
that contains a unique printable value that identifies the thread.
The string should be a single sequence of alphanumeric or underscore
characters, and NULL terminated.
\begin{notes}
ilaguna: Why allowing only alphanumeric or underscore characters? As an 
implementer I may want to use colon or slash characters for more structured 
names.
\end{notes}
\begin{quote}
\begin{lstlisting}
ompd_rc_t ompd_get_thread_handle_string_id (
  ompd_thread_handle_t   *thread_handle,                            /* IN */
  char                  **string_id                                /* OUT */
);
\end{lstlisting}
\end{quote}
\labeldef{get-thread-handle-string-id:def}

The OMPD implementation allocates the string returned in \texttt{*string\_id}
using the allocation routine in the callbacks passed to it
during initialization.
On return the string is owned by the debugger, which is responsible
for deallocating it.

The contents of the strings returned for thread handles
which compare as equal with
\refdef{\texttt{ompd\_thread\_handle\_compare}}{thread-handle-compare:def}
must be the same.

\subsection{Parallel Region Handles}


\paragraph{Retrieve the handle for the innermost parallel region for an OpenMP 
thread.}
The  operation \verb|ompd_get_top_parallel_region|
enables the debugger to obtain a pointer to the parallel handle
for the innermost, or topmost,
parallel region associated with an OpenMP thread.
This call is meaningful only if the thread whose handle is provided is stopped.
 
\begin{quote}
\begin{lstlisting}
ompd_rc_t ompd_get_top_parallel_region (
  ompd_thread_handle_t        *thread_handle,                       /* IN */
  ompd_parallel_handle_t     **parallel_handle                     /* OUT */
);
\end{lstlisting}
\end{quote}
\labeldef{get-top-parallel-region:def}
The parallel handle must be released by calling
\refdef{\texttt{ompd\_release\_parallel\_handle}}{release-parallel-handle:def}.
 
\paragraph{Retrieve the handle for an enclosing parallel region.}
The  \verb|ompd_get_enclosing_parallel_handle|  operation enables
the debugger to obtain a pointer to the parallel handle for the parallel region
enclosing the parallel region specified by \verb|parallel_handle|.
This call is meaningful only if at least one thread in the parallel
region is stopped.

\begin{quote}
\begin{lstlisting}
ompd_rc_t ompd_get_enclosing_parallel_handle (
  ompd_parallel_handle_t        *parallel_handle,                   /* IN */
  ompd_parallel_handle_t       **enclosing_parallel_handle         /* OUT */
);
\end{lstlisting}
\labeldef{get-enclosing-parallel-handle:def}
\end{quote}

On success \texttt{ompd\_get\_enclosing\_parallel\_handle}
returns \texttt{ompd\_rc\_ok}.
A pointer to the parallel handle for the enclosing region is
returned in \texttt{*enclosing\_parallel\_handle}.
After the call the handle is owned by the debugger,
which must release it when it is no longer required by calling
\refdef{\texttt{ompd\_release\_parallel\_handle}}{release-parallel-handle:def}.

\paragraph{Retrieve the handle for the parallel region enclosing a task.}
The  \verb|ompd_get_task_parallel_handle|  operation enables
the debugger to obtain a pointer to the parallel handle for the parallel region
enclosing the task region specified by \verb|task_handle|.
This call is meaningful only if at least one thread in the parallel
region is stopped.

\begin{quote}
\begin{lstlisting}
ompd_rc_t ompd_get_task_parallel_handle (
  ompd_task_handle_t            *task_handle,                       /* IN */
  ompd_parallel_handle_t       **task_parallel_handle              /* OUT */
);
\end{lstlisting}
\end{quote}
\labeldef{get-task-parallel-handle:def}
On success \texttt{ompd\_get\_task\_parallel\_handle}
returns \texttt{ompd\_rc\_ok}.
A pointer to the parallel regions handle
is returned in \texttt{*task\_parallel\_handle}.
The parallel handle is owned by the debugger, which must release it by calling
\refdef{\texttt{ompd\_release\_parallel\_handle}}{release-parallel-handle:def}.

\paragraph{Release a parallel region handle.}
Parallel region handles are opaque to the debugger, which therefore
cannot release them directly.
Instead, when the debugger is finished with a parallel region handle
it must must pass it to the OMPD \texttt{ompd\_release\_parallel\_handle}
routine for disposal.
\begin{quote}
\begin{lstlisting}
ompd_rc_t ompd_release_parallel_handle (
  ompd_parallel_handle_t  *parallel_handle                          /* IN */
);
\end{lstlisting}
\end{quote}
\labeldef{release-parallel-handle:def}

\paragraph{Compare parallel region handles.}
The internal structure of parallel region handles is opaque to the debugger.
While the debugger can easily compare pointers to parallel region handles,
it cannot determine whether handles at two different addresses refer to the
same underlying parallel region.
\begin{quote}
\begin{lstlisting}
ompd_rc_t ompd_parallel_handle_compare (
  ompd_parallel_handle_t  *parallel_handle_1,                       /* IN */
  ompd_parallel_handle_t  *parallel_handle_2,                       /* IN */
  int                     *cmp_value                               /* OUT */
);
\end{lstlisting}
\end{quote}
\labeldef{parallel-handle-compare:def}
On success, \texttt{ompd\_parallel\_handle\_compare} returns in
\texttt{*cmp\_value} a signed integer value that indicates how
the underlying parallel regions compare:
a value less than, equal to, or greater than 0 indicates that
the region corresponding to \texttt{parallel\_handle\_1} is,
respectively, less than, equal to, or greater than that corresponding
to \texttt{parallel\_handle\_2}.

For OMPD implementations that always have a single, unique, underlying
parallel region handle for a given parallel region,
this operation reduces to a simple comparison of the pointers.
However, other implementations may take a different approach,
and therefore the only reliable way of determining whether two different
pointers to parallel regions handles refer the same or distinct
parallel regions is to use \texttt{ompd\_parallel\_handle\_compare}.

Allowing parallel region handles to be compared allows the debugger to hold
them in ordered collections.
The means by which parallel region handles are ordered is
implementation-defined.

\paragraph{String id.}
The \texttt{ompd\_get\_parallel\_handle\_string\_id}
function returns a string that contains a unique printable
value that identifies the parallel region.
The string should be a single sequence of alphanumeric or underscore
characters, and NULL terminated.
\begin{notes}
	ilaguna: Why allowing only alphanumeric or underscore characters? As an 
	implementer I may want to use colon or slash characters for more structured 
	names.
\end{notes}
\begin{quote}
\begin{lstlisting}
ompd_rc_t ompd_get_parallel_handle_string_id (
  ompd_parallel_handle_t   *parallel_handle,                        /* IN */
  char                    **string_id                              /* OUT */
);
\end{lstlisting}
\end{quote}
\labeldef{get-parallel-handle-string-id:def}

The OMPD implementation allocates the string returned in \texttt{*string\_id}
using the allocation routine in the callbacks passed to it
during initialization.
On return the string is owned by the debugger, which is responsible
for deallocating it.

The contents of the strings returned for parallel regions handles
which compare as equal with
\refdef{\texttt{ompd\_parallel\_handle\_compare}}{parallel-handle-compare:def}
must be the same.

\subsection{Task Handles}
\label{task-handles:sec}
\paragraph{Retrieve the handle for the innermost task for an OpenMP thread.}
The  debugger uses the operation  \verb|ompd_get_top_task_region|
to obtain a pointer to the task handle for the innermost, or topmost,
task region associated with an OpenMP thread.
This call is meaningful only if the thread whose handle is provided is stopped.
\begin{quote}
\begin{lstlisting}
ompd_rc_t ompd_get_top_task_region (
  ompd_thread_handle_t        *thread_handle,                       /* IN */
  ompd_task_handle_t         **task_handle                         /* OUT */
);
\end{lstlisting}
\end{quote}
\labeldef{get-top-task-region:def}
The task handle must be released by calling
\refdef{\texttt{ompd\_release\_task\_handle}}{release-task-handle:def}.

\paragraph{Retrieve the handle for an enclosing task.}
The  debugger uses \verb|ompd_get_ancestor_task_region| to obtain
a pointer to the task handle for the task region enclosing the
task region specified by \verb|task_handle|.
This call is meaningful only if the thread executing the task specified by
\verb|task_handle|  is stopped.
\begin{quote}
\begin{lstlisting}
ompd_rc_t  ompd_get_ancestor_task_region (
  ompd_task_handle_t         *task_handle,                          /* IN */
  ompd_task_handle_t        **parent_task_handle                   /* OUT */
);
\end{lstlisting}
\end{quote}
\labeldef{get-ancestor-task-region:def}
The task handle must be released by calling
\refdef{\texttt{ompd\_release\_task\_handle}}{release-task-handle:def}.
 
\paragraph{Retrieve implicit task handle for a parallel region.}
The   \verb|ompd_get_implicit_task_in_parallel|  operation enables
the debugger to obtain a vector of pointers to task handles for all
implicit tasks associated with a parallel region.
This call is meaningful only if all threads associated
with the parallel region are stopped.

\begin{quote}
\begin{lstlisting}
ompd_rc_t ompd_get_implicit_task_in_parallel (
  ompd_parallel_handle_t          *parallel_handle,                 /* IN */
  ompd_task_handle_t            ***task_handle_vector,             /* OUT */
  int                             *num_handles                     /* OUT */
);
\end{lstlisting}
\end{quote}
\labeldef{get-implicit-task-in-parallel:def}
The OMPD must use the memory allocation callback to obtain the
memory for the vector of pointers to task handles returned by the operation.
If the OMPD implementation needs to allocate heap memory for the
task handles it returns, it must use the callbacks to acquire this memory.
After the call the vector and the task handles are `owned' by the debugger,
which is responsible for deallocating them.
The task handles must be released calling
\refdef{\texttt{ompd\_release\_task\_handle}}{release-task-handle:def}.
The vector was allocated by the OMPD implementation using the
allocation routine passed to it during the call to
\refdef{\texttt{ompd\_initialize}}{initialize:def}.
The debugger itself must deallocate the vector in a compatible manner.

\paragraph{Release a task handle.}
Task handles are opaque to the debugger, which therefore cannot release
them directly.
Instead, when the debugger is finished with a task handle it must
pass it to the OMPD \texttt{ompd\_release\_task\_handle} routine
for disposal.
\begin{quote}
\begin{lstlisting}
ompd_rc_t ompd_release_task_handle (
  ompd_task_handle_t  *task_handle                                  /* IN */
);
\end{lstlisting}
\end{quote}
\labeldef{release-task-handle:def}

\paragraph{Compare task handles.}
The internal structure of task handles is opaque to the debugger.
While the debugger can easily compare pointers to task handles,
it cannot determine whether handles at two different addresses refer
to the same underlying task.
\begin{quote}
\begin{lstlisting}
ompd_rc_t ompd_task_handle_compare (
  ompd_task_handle_t  *task_handle_1,                               /* IN */
  ompd_task_handle_t  *task_handle_2,                               /* IN */
  int                 *cmp_value                                   /* OUT */
);
\end{lstlisting}
\end{quote}
\labeldef{task-handle-compare:def}
On success, \texttt{ompd\_task\_handle\_compare} returns in
\texttt{*cmp\_value} a signed integer value that indicates how
the underlying tasks compare:
a value less than, equal to, or greater than 0 indicates that
the task corresponding to \texttt{task\_handle\_1} is,
respectively, less than, equal to, or greater than that corresponding
to \texttt{task\_handle\_2}.

For OMPD implementations that always have a single, unique, underlying
task handle for a given task,
this operation reduces to a simple comparison of the pointers.
However, other implementations may take a different approach,
and therefore the only reliable way of determining whether two different
pointers to task handles refer the same or distinct
task is to use \texttt{ompd\_task\_handle\_compare}.

Allowing task handles to be compared allows the debugger to hold
them in ordered collections.
The means by which task handles are ordered is implementation-defined.

\paragraph{String id.}
The \texttt{ompd\_get\_task\_handle\_string\_id}
function returns a string that contains a unique printable
value that identifies the task.
The string should be a single sequence of alphanumeric or underscore
characters, and NULL terminated.
\begin{notes}
	ilaguna: Why allowing only alphanumeric or underscore characters? As an 
	implementer I may want to use colon or slash characters for more structured 
	names.
\end{notes}
\begin{quote}
\begin{lstlisting}
ompd_rc_t ompd_get_task_handle_string_id (
  ompd_task_handle_t   *task_handle,                                /* IN */
  char                **string_id                                  /* OUT */
);
\end{lstlisting}
\end{quote}
\labeldef{get-task-handle-string-id:def}

The OMPD implementation allocates the string returned in \texttt{*string\_id}
using the allocation routine in the callbacks passed to it
during initialization.
On return the string is owned by the debugger, which is responsible
for deallocating it.

The contents of the strings returned for task handles
which compare as equal with
\refdef{\texttt{ompd\_task\_handle\_compare}}{task-handle-compare:def}
must be the same.