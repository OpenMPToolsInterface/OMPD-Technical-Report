\section{OMPD Interface Type Definitions}
\label{appendix:ompd-types}

The ompd.h file contains declarations and definitions for OMPD API
types, structures, and functions.

\subsection{Basic Types}

\begin{quote}
\begin{lstlisting}

typedef uint64_t ompd_taddr_t;          /* unsigned integer large enough */
                                        /* to hold a target address or a */
                                        /* target segment value          */
typedef int64_t  ompd_tword_t;          /* signed version of ompd_addr_t */
typedef uint64_t ompd_parallel_id_t;    /* parallel region instance ID   */
typedef uint64_t ompd_task_id_t;        /* task region instance ID       */
typedef uint64_t ompd_wait_id_t;        /* identifies what a thread is   */
                                        /* waiting for                   */
typedef struct {
  ompd_taddr_t segment;                 /* target architecture specific  */
                                        /* segment value                 */
  ompd_taddr_t address;                 /* target address in the segment */
} ompd_address_t;

#define OMPD_SEGMENT_UNSPECIFIED  ((ompd_taddr_t) 0)
#define OMPD_SEGMENT_TEXT         ((ompd_taddr_t) 1)
#define OMPD_SEGMENT_DATA         ((ompd_taddr_t) 2)

\end{lstlisting}
\end{quote}

An \verb|ompd_address_t| is a structure that OMPD uses to specify
target addresses, which may or may not be segmented.  The following
rules apply:
\begin{itemize}
%
\item
If the target architecture is not segmented, the OMPD implementation
should use \verb|OMPD_SEGMENT_UNSPECIFIED| for the segment value.
%
\item
If the target architecture uses simple ``text'' and ``data'' segments,
which is common on some systems, the OMPD implementation should use
\verb|OMPD_SEGMENT_TEXT| for the text segment value, and
\verb|OMPD_SEGMENT_DATA| for the data segment value.
%
\item
The segment value for the NVIDIA\textsuperscript{\textregistered} GPU
target architecture should use a \verb|ptxStorageKind| enumeration
value as defined by the CUDA Debugger API.
%
This enumeration is defined by the \verb|cudadebugger.h| header file
contained within a CUDA SDK package.
%
%% Included for reference. Not sure if this should be part of the OMPD spec.
%% 
%% /*--------------------- Memory Segments (as used in DWARF) 
%%-------------------*/
%% 
%% typedef enum {
%%     ptxUNSPECIFIEDStorage,
%%     ptxCodeStorage,
%%     ptxRegStorage,
%%     ptxSregStorage,
%%     ptxConstStorage,
%%     ptxGlobalStorage,
%%     ptxLocalStorage,
%%     ptxParamStorage,
%%     ptxSharedStorage,
%%     ptxSurfStorage,
%%     ptxTexStorage,
%%     ptxTexSamplerStorage,
%%     ptxGenericStorage,
%%     ptxIParamStorage,
%%     ptxOParamStorage,
%%     ptxFrameStorage,
%%     ptxMAXStorage
%% } ptxStorageKind;
%
\item
Otherwise, the segment value is target architecture specific.
\end{itemize}


\subsection{Operating System Thread Information}

An OpenMP runtime may be implemented on different threading substrates.
OMPD uses the \verb|ompd_osthread_kind_t| type to describe an operating
system thread upon which an OpenMP thread is overlaid.

\begin{quote}
\begin{lstlisting}
typedef enum {
  ompd_osthread_pthread,
  ompd_osthread_lwp,
  ompd_osthread_winthread
} ompd_osthread_kind_t;
\end{lstlisting}
\labeldef{osthread-kind-t:def}
\end{quote}

The operating system-specific information can vary in size
and format, and therefore is not explicitly represented in this API.
Operating system-specific thread identifiers are passed
across the interface by reference, that is, by a pointer
to where the information can be found.
In addition, the `kind' and size of the information are
also passed.

When operating system-specific thread identifiers are
passed as either `in' or `out' parameters, they are
allocated and owned by the caller, which is responsible
for their eventual disposal.

\subsection{OMPD Handles}
\label{ompd-handles:sec}

Each OMPD interface operation that applies to a particular address space,
thread, parallel region, or task must  explicitly specify the target entity
for the operation using a \emph{handle}.
OMPD employs handles for address spaces (for a host or target device),
threads, parallel regions, and tasks.
A handle for an entity is constant while the entity itself is live.
Handles are defined by the OMPD implementation, and are opaque to the
debugger.
This is how the \texttt{ompd.h} header file defines these types:

\begin{quote}
\begin{lstlisting}
typedef struct _ompd_address_space_handle_s  ompd_address_space_handle_t;
typedef struct _ompd_thread_handle_s         ompd_thread_handle_t;
typedef struct _ompd_parallel_handle_s       ompd_parallel_handle_t;
typedef struct _ompd_task_handle_s           ompd_task_handle_t;
\end{lstlisting}
\end{quote}

Defining the externally visible type names in this way introduces
an element of type safety to the interface, and will help to catch
instances where incorrect handles are passed by the debugger
to the OMPD implementation.
The \texttt{struct}s do not need to be defined at all.
The OMPD implementation would need to cast incoming (pointers to)
handles to the appropriate internal, private types.

\subsection{Debugger Contexts}
\label{debugger-contexts:sec}

The debugger contexts are opaque to the OMPD, and are defined
in the \texttt{ompd.h} header file as follows:

%% \begin{comment} %by Joachim
%% \begin{quote}
%% \begin{lstlisting}
%% typedef struct _ompd_process_context_s        ompd_process_context_t;
%% typedef struct _ompd_address_space_context_s  ompd_address_space_context_t;
%% typedef struct _ompd_thread_context_s         ompd_thread_context_t;
%% \end{lstlisting}
%% \end{quote}
%% \end{comment}

\begin{quote}
\begin{lstlisting}
typedef struct _ompd_address_space_context_s  ompd_address_space_context_t;
typedef struct _ompd_thread_context_s         ompd_thread_context_t;
\end{lstlisting}
\end{quote}

\subsection{Return Codes}
Each OMPD interface operation has a return code.
The purpose of the each return code is explained
by the comments in the definition below.
\begin{quote}
\begin{lstlisting}
typedef enum {
  ompd_rc_ok            =  0, /* operation was successful                 */
  ompd_rc_unavailable   =  1, /* info is not available (in this context)  */
  ompd_rc_stale_handle  =  2, /* handle is no longer valid                */
  ompd_rc_bad_input     =  3, /* bad input parameters (other than handle) */
  ompd_rc_error         =  4, /* error                                    */
  ompd_rc_unsupported   =  5, /* operation is not supported               */
  ompd_rc_needs_state_tracking =  6,
                              /* needs runtime state tracking enabled     */
  ompd_rc_incompatible  =  7, /* target is not compatible with this OMPD  */
  ompd_rc_target_read_error =  8,
                              /* error reading from the target            */
  ompd_rc_target_write_error =  9,
                              /* error writing from the target            */
  ompd_rc_nomem         = 10, /* unable to allocate memory                */
} ompd_rc_t;
\end{lstlisting}
\labeldef{rc-t:def}
\end{quote}

%%%%%%%%%%%%%%%%%%%%%%%%%%%%%%%%%%%%%%%%%%%%%%%%%%%%%%%%%%%%%%%%%%%%%%
%
\subsection{OpenMP Scheduling}
\labeldef{openmp-scheduling:def}
%
This enumeration defines \verb|ompd_sched_t|, which is the OMPD API
definition corresponding to the OpenMP enumeration type
\verb|omp_sched_t| (\S3.2.12 of~\cite{OpenMP}).
%
\verb|ompd_sched_t| also defines \verb|ompd_sched_vendor_lo| and
\verb|ompd_sched_vendor_hi| to define the range of
implementation-specific \verb|omp_sched_t| values than can be handle
by the OMPD API.
%
\begin{quote}
\begin{lstlisting}
typedef enum {
  ompd_sched_static = 1,
  ompd_sched_dynamic = 2,
  ompd_sched_guided = 3,
  ompd_sched_auto = 4,
  ompd_sched_vendor_lo = 5,
  ompd_sched_vendor_hi = 0x7fffffff
} ompd_sched_t;
\end{lstlisting}
\end{quote}

%%%%%%%%%%%%%%%%%%%%%%%%%%%%%%%%%%%%%%%%%%%%%%%%%%%%%%%%%%%%%%%%%%%%%%
%
\subsection{OpenMP Proc Binding}
\labeldef{openmp-proc-binding:def}
%
This enumeration defines \verb|ompd_proc_bind_t|, which is the OMPD
API definition corresponding to the OpenMP enumeration type
\verb|omp_proc_bind_t| (\S3.2.22 of~\cite{OpenMP}).
%
\begin{quote}
\begin{lstlisting}
typedef enum {
  ompd_proc_bind_false = 0,
  ompd_proc_bind_true = 1,
  ompd_proc_bind_master = 2,
  ompd_proc_bind_close = 3,
  ompd_proc_bind_spread = 4
} ompd_proc_bind_t;
\end{lstlisting}
\end{quote}

%%%%%%%%%%%%%%%%%%%%%%%%%%%%%%%%%%%%%%%%%%%%%%%%%%%%%%%%%%%%%%%%%%%%%%
%
\subsection{Primitive Types}
%
This structure contains members that the OMPD implementation can use
to interrogate the debugger about the ``sizeof'' of primitive types in
the target address space.
%
\begin{quote}
\begin{lstlisting}
typedef struct {
  int sizeof_char;
  int sizeof_short;
  int sizeof_int;
  int sizeof_long;
  int sizeof_long_long;
  int sizeof_pointer;
} ompd_target_type_sizes_t;
\end{lstlisting}
\end{quote}

This enumeration of primitive types is used by OMPD to express the primitive
type of data for target to host conversion.

\begin{quote}
\begin{lstlisting}
typedef enum {
  ompd_type_char        = 0,
  ompd_type_short       = 1,
  ompd_type_int         = 2,
  ompd_type_long        = 3,
  ompd_type_long_long   = 4,
  ompd_type_pointer     = 5
} ompd_target_prim_types_t;
\end{lstlisting}
\end{quote}

\subsection{Runtime States}

The OMPD runtime states mirror those in OMPT (see Appendix~A
of~\cite{ompt-tr2}).

\begin{quote}
\begin{lstlisting}
typedef enum {
  /* work states (0..15) */
  ompd_state_work_serial           = 0x00,   /* working outside parallel */
  ompd_state_work_paralle l        = 0x01,   /* working within parallel  */
  ompd_state_work_reduction        = 0x02,   /* performing a reduction   */

  /* idle (16..31) */
  ompd_state_idle                  = 0x10,   /* waiting for work         */

  /* overhead states (32..63) */
  ompd_state_overhead              = 0x20,   /* non-wait overhead        */

  /* barrier wait states (64..79) */
  ompd_state_wait_barrier          = 0x40,   /* generic barrier          */
  ompd_state_wait_barrier_implicit = 0x41,   /* implicit barrier         */
  ompd_state_wait_barrier_explicit = 0x42,   /* explicit barrier         */

  /* task wait states (80..95) */
  ompd_state_wait_taskwait         = 0x50,   /* waiting at a taskwait    */
  ompd_state_wait_taskgroup        = 0x51,   /* waiting at a taskgroup   */

  /* mutex wait states (96..111) */
  ompd_state_wait_lock             = 0x60,   /* waiting for lock         */
  ompd_state_wait_nest_lock        = 0x61,   /* waiting for nest lock    */  
  ompd_state_wait_critical         = 0x62,   /* waiting for critical     */
  ompd_state_wait_atomic           = 0x63,   /* waiting for atomic       */
  ompd_state_wait_ordered          = 0x64,   /* waiting for ordered      */

  /* misc (112..127) */
  ompd_state_undefined             = 0x70,   /* undefined thread state    */
  ompd_state_first                 = 0x71,   /* initial enumeration state */
} ompd_state_t;
\end{lstlisting}
\end{quote}
\labeldef{state-t:def}

\subsection{Type Signatures for Debugger Callbacks}
For OMPD to provide information  about the internal state
of the OpenMP runtime system in a target process or core file,
it must have a means to extract information from
the target.
A target ``process'' may be a ``live'' process or a core file. 
A target thread may be a ``live'' thread in a process, or a thread in a core 
file.
To enable OMPD to extract state information from a target process or core file,
a debugger supplies OMPD with callback functions to inquire
about the size of primitive types in the target,
look up the addresses of symbols,
as well as read and write memory in the target.
OMPD then uses these callbacks to implement its interface operations.
Signatures for the debugger callbacks used by OMPD are given below.

\paragraph{Memory management.}
The callback signatures below are used to allocate and free
memory in the debugger's address space.
The OMPD DLL \emph{must} obtain and release heap memory \emph{only}
using the callbacks provided to it by the debugger.
It must \emph{not} call the heap manager directly using \texttt{malloc}.
For C++ implementations this means the OMPD implementation \emph{must}
overload the functions \texttt{new}, 
\texttt{new(throw)}, \texttt{new[]}, \texttt{delete}, \texttt{delete(throw)}, 
and \texttt{delete[]} in \emph{all} their variants and use the
debugger-provided callback functions to implement them.
\begin{quote}
\begin{lstlisting}
typedef ompd_rc_t (*ompd_dmemory_alloc_fn_t) (
  ompd_size_t bytes,               /* IN: the number of bytes to allocate */
  void   **ptr /* OUT: on success, a pointer to the allocated memory here */
);

typedef ompd_rc_t (*ompd_dmemory_free_fn_t) (
  void *ptr            /* IN: the address of the memory to be deallocated */
);
\end{lstlisting}
\end{quote}

\paragraph{Context management.}
The callback signature below is used to map an operating system thread
handle to a debugger thread context. 
The OMPD implementation can use this thread context to access thread
local storage (TLS).
\begin{quote}
\begin{lstlisting}
typedef ompd_rc_t (*ompd_get_thread_context_for_osthread_fn_t) (
  ompd_address_space_context_t  *address_space_context,             /* IN */
  ompd_osthread_kind_t           kind,                              /* IN */
  ompd_size_t                    sizeof_osthread,                   /* IN */
  const void                    *osthread,                          /* IN */
  ompd_thread_context_t        **thread_context                    /* OUT */
);
\end{lstlisting}
\end{quote}
On success, the \texttt{ompd\_thread\_context\_t} corresponding
to the operating system thread identifier \texttt{*osthread}
of type \texttt{kind} and size \texttt{sizeof\_osthread} is returned
in \texttt{*thread\_context}.
The thread context is created, and remains owned, by the debugger.
The OMPD implementation can assume that the thread context is valid
for as long as the debugger is holding any references to thread handles
that may contain the thread context.
%%% The OMPD implementation must release the thread context when
%%% it is no longer required by calling the \texttt{release\_thread\_context}
%%% callback.

\paragraph{Context navigation.}
The following callback signature is used to ``navigate'' address space
and thread object relationships.

{\bf Thread context to address space context.}
Given a thread context, get the address space context for the thread
and return it in \verb|*address_space_context|. 
If \verb|thread_context| refers to a host device thread, this function
returns the context for the host address space.
If \verb|thread_context| refers to a target device thread, this
function returns the context for the target device's address space.
\begin{quote}
\begin{lstlisting}
typedef ompd_rc_t (*ompd_get_address_space_context_for_thread_fn_t) (
  ompd_thread_context_t         *thread_context,                   /* IN */
  ompd_address_space_context_t **address_space_context            /* OUT */
);
\end{lstlisting}
\end{quote}

%% \begin{comment} %by Joachim
%% {\bf Process context to host device address space context.}
%% Given a process context, get the address space context for the host
%% address space and return it in \verb|*address_space_context|. 
%% \begin{quote}
%% \begin{lstlisting}
%% typedef ompd_rc_t (*ompd_get_address_space_context_for_process_fn_t) (
%%   ompd_process_context_t        *process_context,                   /* IN */
%%   ompd_address_space_context_t **address_space_context             /* OUT */
%% );
%% \end{lstlisting}
%% \end{quote}
%% 
%% {\bf Thread context to process context.}
%% Given a thread context, get the process context for the process
%% containing the thread and return it in \verb|*process_context|.
%% \begin{quote}
%% \begin{lstlisting}
%% typedef ompd_rc_t (*ompd_get_process_context_for_thread_fn_t) (
%%   ompd_thread_context_t         *thread_context,                    /* IN */
%%   ompd_process_context_t       **process_context                   /* OUT */
%% );
%% \end{lstlisting}
%% \end{quote}
%% 
%% {\bf Address space context to process context.}
%% Given an address space context, get the process context for the
%% process containing the address space and return it in
%% \verb|*process_context|.
%% \begin{quote}
%% \begin{lstlisting}
%% typedef ompd_rc_t (*ompd_get_process_context_for_address_space_fn_t) (
%%   ompd_address_space_context_t  *address_space_context,             /* IN */
%%   ompd_process_context_t       **process_context                   /* OUT */
%% );
%% \end{lstlisting}
%% \end{quote}
%% \end{comment}


\paragraph{Primitive type size.}
The callback signature below is used to look up the sizes
of primitive types in the target address space.
\begin{quote}
\begin{lstlisting}
typedef ompd_rc_t (*ompd_tsizeof_prim_fn_t) (
  ompd_address_space_context_t  *context,                           /* IN */
  ompd_target_type_sizes_t *sizes             /* OUT: returned type sizes */
);
\end{lstlisting}
\end{quote}

\paragraph{Symbol lookup.}
The callback signature below is used to look up
the address of a global symbol in the target.
The argument \texttt{thread\_context} is optional for global memory access
and is NULL in this case. 
If the \texttt{thread\_context} argument is not NULL,
this will give the thread specific context for the symbol lookup,
for the purpose of calculating thread local storage (TLS) addresses.
\begin{quote}
\begin{lstlisting}
typedef ompd_rc_t (*ompd_tsymbol_addr_fn_t) (
  ompd_address_space_context_t *address_space_context,              /* IN */
  ompd_thread_context_t        *thread_context, /* IN: TLS thread or NULL */
  const char                   *symbol_name,    /* IN: global symbol name */
  ompd_address_t               *symbol_addr           /* OUT: on success, */
                                                    /* the symbol address */
);
\end{lstlisting}
\end{quote}
The symbol name supplied by the OMPD implementation is used verbatim
by the debugger, and in particular, no name mangling is performed
prior to the lookup.

%% \begin{comment}
%% \paragraph{Type lookup.} The callback signature below is used to look up a 
%%type in the target.
%% \begin{lstlisting}
%% typedef ompd_rc_t (*ompd_ttype_fn_t) (
%%   ompd_context_t *context,  /* debugger handle for the target */
%%   const char *type_name,    /* name of the type/structure */
%%   ompd_ttype_handle_t *ttype_handle  /* a successful call returns the type 
%%handle here */
%% );
%% \end{lstlisting}

%% \paragraph{Type size lookup.} The callback signature below is used to look 
%%up the size of a type in the target.
%% \begin{lstlisting}
%% typedef ompd_rc_t (*ompd_ttype_sizeof_fn_t) (
%%   ompd_context_t *context,  /* debugger handle for the target */
%%   ompd_ttype_handle_t *ttype_handle,    /* handle of the type/structure */
%%   ompd_tword_t *type_size         /* a successful call returns the type size 
%%here */
%% );
%% \end{lstlisting}

%% \paragraph{Type field offset lookup.} The callback signature below is used 
%%to look up the offset of a field in a type in the target.
%% \begin{lstlisting}
%% typedef ompd_rc_t (*ompd_ttype_offset_fn_t) (
%%   ompd_context_t *context,  /* debugger handle for the target */
%%   ompd_ttype_handle_t *ttype_handle,    /* handle of the type/structure */
%%   const char *field_name,   /* field of interest in the type/structure */
%%   ompd_tword_t *field_offset     /* a successful call returns the field 
%%offset here */
%% );
%% \end{lstlisting}
%% \end{comment}

\paragraph{Memory access.}
The callback signatures below are used to read or write memory in the target.
Data transfers are of unstructured bytes; it is the responsibility
of the OMPD implementation to arrange for any byte swapping as necessary.
The argument \texttt{thread\_context} is optional for global memory access
and is NULL in this case. 
If the argument is not NULL, it identifies the thread specific context
for the memory access, for the purpose of accessing thread local
storage (TLS) memory.
The buffer is allocated and owned by the OMPD implementation.
\begin{quote}
\begin{lstlisting}
typedef ompd_rc_t (*ompd_tmemory_read_fn_t) (
  ompd_address_space_context_t *address_space_context,              /* IN */
  ompd_thread_context_t        *thread_context, /* IN: TLS thread or NULL */
  const ompd_address_t         *addr,        /* IN: address in the target */
  ompd_tword_t                  nbytes,      /* IN: number of bytes to be */
                                                           /* transferred */
  void                         *buffer  /* OUT: buffer for data read from */
                                                            /* the target */
);

typedef ompd_rc_t (*ompd_tmemory_write_fn_t) (
  ompd_address_space_context_t *address_space_context,              /* IN */
  ompd_thread_context_t        *thread_context, /* IN: TLS thread or NULL */
  const ompd_address_t         *addr,        /* IN: address in the target */
  ompd_tword_t                  nbytes,      /* IN: number of bytes to be */
                                                          /* transferred  */
  const void                   *buffer  /* IN: buffer for date written to */
                                                            /* the target */
);
\end{lstlisting}
\end{quote}

\paragraph{Data format conversion.}
The callback signature below is used to convert data from the target 
address space byte ordering to the host (OMPD implementation) byte ordering,
and vice versa.

\begin{quote}
\begin{lstlisting}
typedef ompd_rc_t (*ompd_target_host_fn_t) (
  ompd_address_space_context_t *address_space_context,              /* IN */
  const void                   *input,                              /* IN */
  int                           unit_size,                          /* IN */
  int                           count,    /* IN: number of primitive type */
                                                      /* items to process */
  void                         *output                             /* OUT */
);
\end{lstlisting}
\end{quote}
The input and output buffers are allocated and owned by the
OMPD implementation, and it is its responsibility to ensure
that the buffers are the correct size.


%% \begin{comment}
%% \paragraph{Get error string.}
%% The callback signature below is used by OMPD to retrieve
%% an error string from the debugger given an error code.


%% \begin{lstlisting}
%% typedef ompd_rc_t (*ompd_error_string_fn_t) (
%%   ompd_context_t  *context,         /* IN: debugger handle for the target */
%%   int              error_code,
%%   const char     **string
%% \end{lstlisting}
%% \end{comment}

\paragraph{Print string.}
The callback signature below is used by OMPD to have the debugger
print a string. OMPD should not print directly. 
\begin{quote}
\begin{lstlisting}
typedef ompd_rc_t (*ompd_print_string_fn_t) (
  const char              *string
);
\end{lstlisting}
\end{quote}