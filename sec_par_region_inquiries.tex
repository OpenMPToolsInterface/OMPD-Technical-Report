\section{Parallel Region Inquiries}

We describe OMPD functions to perform inquiries about parallel regions.

\subsection{Parallel Region Settings}

\paragraph{Determine the number of threads associated with a parallel region.}

\begin{quote}
\begin{lstlisting}
ompd_rc_t ompd_get_num_threads (
  ompd_parallel_handle_t  *parallel_handle,                         /* IN */
  ompd_tword_t            *val                                     /* OUT */
);  
\end{lstlisting}
\end{quote}
\labeldef{get-num-threads:def}

\paragraph{Determine the nesting depth of a particular parallel region.}

\begin{quote}
\begin{lstlisting}
ompd_rc_t ompd_get_level (
  ompd_task_handle_t  *task_handle,                                 /* IN */
  ompd_tword_t        *val                                         /* OUT */
); 
\end{lstlisting}%
\end{quote}
\labeldef{get-level:def}

\paragraph{Determine the number of enclosing  parallel regions.}
\verb|ompd_get_active_level| returns the number of nested, active
parallel regions enclosing the parallel region specified by its handle.

\begin{quote}
\begin{lstlisting}
ompd_rc_t ompd_get_active_level (
  ompd_task_handle_t  *task_handle,                                 /* IN */
  ompd_tword_t        *val                                         /* OUT */
); 
\end{lstlisting}
\end{quote}
\labeldef{get-active-level:def}

\subsection{OMPT Parallel Region Inquiry Analogues}

\paragraph{Parallel Function Entry Point}

The \texttt{ompd\_get\_parallel\_function} returns the
entry point of the code that corresponds to the body of
the parallel construct.

\begin{quote}
\begin{lstlisting}
ompd_rd_t ompd_get_parallel_function (
  ompd_parallel_handle_t  *parallel_handle,                         /* IN */
  ompd_address_t          *entry_point                             /* OUT */
);
\end{lstlisting}
\end{quote}
\labeldef{get-parallel-function:def}