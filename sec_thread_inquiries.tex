\section{Thread Inquiries}

We describe OMPD functions to perform inquiries about threads.

\subsection{Operating System Thread Inquiry}

\paragraph{Mapping an operating system thread to an OMPD thread handle.}
OMPD provides the function \verb|ompd_get_thread_handle|
to inquire whether an operating system thread is an OpenMP
thread or not.
If the function returns \verb|ompd_rc_ok|, then the operating
system thread is an OpenMP thread and \verb|*thread_handle|
will be initialized to a pointer to the thread handle for
the OpenMP thread.

\begin{quote}
\begin{lstlisting}
ompd_rc_t ompd_get_thread_handle (
  ompd_address_space_handle_t   *handle,                            /* IN */
  ompd_osthread_kind_t           kind,                              /* IN */
  ompd_size_t                    sizeof_osthread,                   /* IN */
  const void                    *osthread,                          /* IN */
  ompd_thread_handle_t         **thread_handle                     /* OUT */
);
\end{lstlisting}
\end{quote}
\labeldef{get-thread-handle:def}
The operating system ID \texttt{*osthread} is guaranteed
to be valid for the duration of the call.
If the OMPD implementation needs to retain the operating system-specific
thread identifier it must copy it.

The thread handle \texttt{*thread\_handle} returned by the OMP implementation
is `owned' by the debugger, which must release it by calling
\refdef{\texttt{ompd\_release\_thread\_handle}}{release-thread-handle:def}.
If \texttt{os\_thread} does not refer to an OpenMP thread,
\texttt{ompd\_get\_thread\_handle} returns \texttt{ompd\_rc\_bad\_input}
and \texttt{*thread\_handle} is also set to NULL.

\paragraph{Mapping an OMPD thread handle to an operating system thread.}
\verb|ompd_get_osthread| performs the mapping between an OMPD
thread handle and an operating system-specific thread identifier.
\begin{quote}
\begin{lstlisting}
ompd_rc_t ompd_get_osthread (
  ompd_thread_handle_t  *thread_handle,                             /* IN */
  ompd_osthread_kind_t   kind,                                      /* IN */
  ompd_size_t            sizeof_osthread,                           /* IN */
  void                  *osthread                                  /* OUT */
);
\end{lstlisting}
\end{quote}
\labeldef{get-osthread:def}

The caller indicates what \emph{kind} of operating system-specific thread
identifier it wants by setting the \refdef{\texttt{kind}}{osthread-kind-t:def}
`in' parameter.
It also passes a pointer to the buffer into which the OMPD
implementation writes the operating system-specific thread identifier,
and the size of the buffer, to the OMPD implementation.
The buffer is owned by the debugger.

On success \texttt{ompd\_get\_osthread} returns \texttt{rc\_ok},
and returns the operating system-specific thread identifier in
\texttt{*osthread}.
If the operation fails, the OMPD implementation returns
the appropriate value from \refdef{\texttt{ompd\_rc\_t}}{rc-t:def}.
Note that the operation should fail if the OMPD implementation is
unable to return an operating system-specific identifier of the
requested `kind' or size.


\subsection{Thread State Inquiry Analogue}

The function \verb|ompd_get_state| is a  third-party version of
\verb|ompt_get_state|. 
The only difference between the OMPD and OMPT counterparts
is that the OMPD version must supply a thread handle to provide
a context for this inquiry.
% Calls to this function need not return meaningful results unless the thread 
%is stopped.

\begin{quote}
\begin{lstlisting}
ompd_rc_t ompd_get_state (
  ompd_thread_handle_t  *thread_handle,                             /* IN */
  ompd_state_t          *state,                                    /* OUT */
  ompd_wait_id_t        *wait_id                                   /* OUT */
);
\end{lstlisting}
\end{quote}
