\section{Task Inquiries}

We describe OMPD functions to perform inquiries about tasks.

\subsection{Task Function Entry Point}

The \verb|ompd_get_task_function| returns the entry point of the code
that corresponds to the body of code executed by the task:
\begin{quote}
\begin{lstlisting}
ompd_rc_t ompd_get_task_function (
  ompd_task_handle_t  *task_handle,                                 /* IN */
  ompd_address_t      *entry_point                                 /* OUT */
);
\end{lstlisting}
\end{quote}
\labeldef{get-task-function:def}

\subsection{Task Settings}

Here we describe functions to retrieve information from OpenMP tasks, including 
the values of some \emph{Internal Control Variables (ICVs)}.
A target is able to get the information defined here directly from
the runtime. For this reason, these inquiry functions have no counterparts in
the OMPT interface. The only difference between the OMPD inquiry operations and 
their counterparts in the OpenMP runtime is that the OMPD version
must supply a task handle to provide a context for each inquiry.
Values are returned through the `out' parameter \texttt{val}.

The \verb|ompd_get_max_threads| function returns the value of the
target's \emph{nthreads-var} ICV (\S2.3.1 of~\cite{OpenMP}),
and corresponds to the \verb|omp_get_max_threads| function
in the OpenMP runtime API.
This returns an upper bound on the number threads that could be used
to form a new team if a \texttt{parallel} construct without a
\texttt{num\_threads} clause were encountered.
\begin{quote}
\begin{lstlisting}
ompd_rc_t ompd_get_max_threads (
  const ompd_task_handle_t  *task_handle,                           /* IN */
  ompd_tword_t              *val                                   /* OUT */
); 
\end{lstlisting}
\end{quote}
The \emph{nthreads-var} ICV is defined in OpenMP as a list
(\S2.3.2 of~\cite{OpenMP}).
Like \verb|omp_get_max_threads|, \verb|ompd_get_max_threads|
returns the first element of the list.
\begin{notes}
ilaguna: why the first element if the function is named 'max'? This could 
confuse readers.
\end{notes}

\begin{quote}
\begin{lstlisting}
ompd_rc_t ompd_get_thread_num (
  const ompd_thread_handle_t  *thread_handle,                       /* IN */
  ompd_tword_t                *val                                 /* OUT */
);
\end{lstlisting}
\end{quote}
\verb|ompd_get_thread_num| corresponds to the \verb|omp_get_thread_num|
routine in the OpenMP runtime, and returns the thread's logical
thread number in the team.
 
\verb|ompd_in_parallel| returns logical true (\textit{i.e.}, \texttt{*val != 0})
if \emph{active-levels-var}
ICV (\S2.3.1 of~\cite{OpenMP}) is greater than~0, and false (0) otherwise.
The routine corresponds to \verb|omp_in_parallel| in the OpenMP runtime.
\begin{quote}
\begin{lstlisting}
ompd_rc_t ompd_in_parallel (
  const ompd_task_handle_t  *task_handle,                           /* IN */
  ompd_tword_t              *val                                   /* OUT */
); 
\end{lstlisting}
\end{quote}

\verb|ompd_in_final| corresponds to \verb|omp_in_final| and returns
logical true if the task is a final task.
\begin{quote}
\begin{lstlisting}
ompd_rc_t ompd_in_final (
  const ompd_task_handle_t  *task_handle,                           /* IN */
  ompd_tword_t              *val                                   /* OUT */
); 
\end{lstlisting}
\end{quote}

\verb|ompd_get_dynamic| returns the value of the
\emph{dyn-var} ICV (\S2.3.1 of~\cite{OpenMP}),
and corresponds to the \verb|omp_get_dynamic| member of the OpenMPI API.
\begin{quote}
\begin{lstlisting}
ompd_rc_t ompd_get_dynamic (
  const ompd_task_handle_t  *task_handle,                           /* IN */
  ompd_tword_t              *val                                   /* OUT */
);
\end{lstlisting}
\end{quote}
\emph{dyn-var}~determines whether dynamic adjustment of the number
of threads is enabled or disabled.

\verb|ompd_get_nested| corresponds to \verb|omp_get_nested|,
and returns the value of the \emph{nest-var} ICV (\S2.3.1 of~\cite{OpenMP}).
\begin{quote}
\begin{lstlisting}
ompd_rc_t ompd_get_nested (
  const ompd_task_handle_t  *task_handle,                           /* IN */
  ompd_tword_t              *val                                   /* OUT */
 );
\end{lstlisting}
\end{quote}
\emph{nest-var} determines if nested parallelism is enabled;
a logical true value indicates that it is, false that it is not.
 
The maximum number of nested levels parallelism is returned by
\verb|get_max_active_levels|.
\begin{quote}
\begin{lstlisting}
ompd_rc_t ompd_get_max_active_levels (
  const ompd_thread_handle_t  *thread_handle,                       /* IN */
  ompd_tword_t                *val                                 /* OUT */
);
\end{lstlisting}
\end{quote}
This operation corresponds to the OpenMP routine
\verb|omp_get_max_active_levels|
and the ICV \emph{max-active-levels-var} (\S2.3.1 of~\cite{OpenMP}).%

\begin{notes}
Ariel: I think this may need a little attention.
What is the scope of this operation?  The OpenMP4 docment refers
to a device.

John: The OpenMP spec leaves ``device'' kind of vague.
The glossary says: ``An implementation defined logical execution engine.
COMMENT: A device could have one or more processors.''
And to a certain extent, I'm not sure it matters to OMPD.
``3.2.16 \texttt{omp\_get\_max\_active\_levels}'' in the OpenMP spec
implies that a thread is required, which is all I think OMPD
needs to care about.

Ariel:
I suppose that the thread has a device associated with it.
\end{notes}

%%%%%%%%%%%%%%%%%%%%%%%%%%%%%%%%%%%%%%%%%%%%%%%%%%%%%%%%%%%%%%%%%%%%%%
%
\verb|ompd_get_schedule| returns information about the schedule that is
applied when \verb|runtime| scheduling is used.
%
This information is represented in the target by the
\emph{run-sched-var} ICV (\S3.2.1 of~\cite{OpenMP}).
%
\begin{quote}
\begin{lstlisting}
ompd_rc_t ompd_get_schedule (
  ompd_task_handle_t        *task_handle,                           /* IN */
  ompd_sched_t              *kind,                                 /* OUT */
  ompd_tword_t              *modifier                              /* OUT */
); 
\end{lstlisting}
\end{quote}
%
OpenMP defines a minimum set of values in the enumeration type
\verb|omp_sched_t| (\S3.2.12 of~\cite{OpenMP}).
%
The OMPD API defines
\refdef{\texttt{ompd\_sched\_t}}{openmp-scheduling:def}, which contains the
corresponding OpenMP enumeration values and ``lo'' and ``hi'' values
for the range of implementation-specific scheduling values that can be
represented by the OMPD API.
%
The scheduling kind is returned in \verb|*kind|.
%
The interpretation of \verb|*modifier| depends on the value of
\verb|*kind|.
%
See \S3.2.12 and \S3.2.13 of~\cite{OpenMP} for further details.

%%%%%%%%%%%%%%%%%%%%%%%%%%%%%%%%%%%%%%%%%%%%%%%%%%%%%%%%%%%%%%%%%%%%%%
%
\verb|ompd_get_proc_bind| returns the value of the task's
\emph{bind-var} ICV (\S2.3.1 of~\cite{OpenMP}), which ``controls the
binding of the OpenMP threads to places,'' or ``default thread
affinity policies.''
%
\begin{quote}
\begin{lstlisting}
ompd_rc_t ompd_get_proc_bind (
  ompd_task_handle_t        *task_handle,                           /* IN */
  ompd_proc_bind_t          *bind                                  /* OUT */
); 
\end{lstlisting}
\end{quote}
The OMPD API defines
\refdef{\texttt{ompd\_proc\_bind\_t}}{openmp-proc-binding:def}, which
contains the corresponding OpenMP enumeration values.
%
The binding is returned in \verb|*bind|.
%
See \S3.2.22 of~\cite{OpenMP} for further details.

\verb|ompd_is_implicit| returns logical true (\textit{i.e.}, \texttt{*val != 0})
if a task is implicit, and false (0) otherwise.
The routine has no corresponding call in the OpenMP runtime.
\begin{quote}
\begin{lstlisting}
ompd_rc_t ompd_is_implicit (
  ompd_task_handle_t  *task_handle,                                 /* IN */
  ompd_tword_t        *val                                         /* OUT */
); 
\end{lstlisting}
\end{quote}


%%%%%%%%%%%%%%%%%%%%%%%%%%%%%%%%%%%%%%%%%%%%%%%%%%%%%%%%%%%%%%%%%%%%%%
%
\subsection{OMPT Task Inquiry Analogues}
%
The function \verb|ompd_get_task_frame|
ia a third-party versions of \verb|ompt_get_task_frame|.
%
\texttt{ompd\_get\_task\_frame} is discussed under Stack Unwinding in
Section~\ref{stack-unwinding:sec}.
%
\begin{quote}
\begin{lstlisting}
ompd_rc_t ompd_get_task_frame (
  ompd_task_handle_t        *task_handle,                           /* IN */
  ompd_address_t            *exit_runtime_addr,                    /* OUT */
  ompd_address_t            *reenter_runtime_addr                  /* OUT */
);
\end{lstlisting}
\end{quote}
\labeldef{get-task-frame:def}
%

%%%%%%%%%%%%%%%%%%%%%%%%%%%%%%%%%%%%%%%%%%%%%%%%%%%%%%%%%%%%%%%%%%%%%%
%
\subsection{Stack Unwinding}
\label{stack-unwinding:sec}
%
\begin{notes}
JVD: This section needs careful review by the OpenMP Tools Working
Group to ensure its correctness.  It depends on whether or not John
Mellor-Crummey's 07/16/15 email proposal to omp-tools@openmp.org to
change the semantics of the \verb|reenter_runtime_addr| field is
adopted.  What we decide, OMPD and OMPT should be consistent.
\end{notes}

The \verb|ompd_get_task_frame| function returns stack frame
information about the target thread associated with the task.
%
This routine corresponds to \verb|ompt_get_task_frame| in the OMPT
API, and the approach for stack inspection is similar to that
described in Appendix~B of~\cite{ompt-tr2}.
%
The \verb|exit_runtime_addr| gives the address of the frame at which
the thread \emph{left} the OpenMP runtime to execute the user code
associated with the task.
%
The \verb|reenter_runtime_addr| is the address of the frame that
called the OpenMP runtime.
\begin{notes}
JVD: Follows John Mellor-Crummey's
07/16/15 email proposal to omp-tools@openmp.org to change the
semantics of the \verb|reenter_runtime_addr| field.)
\end{notes}
%
The debugger can unwind a thread's stack by getting the
thread's current task using
\refdef{\texttt{ompd\_get\_top\_task\_region}}{get-top-task-region:def}.
\begin{notes}
JVD: This assumes that the thread is ``bound'' to the task
handle. Is that correct?
\end{notes}
%
Using the task handle, the debugger can find the thread's exit and
reentry stack frame addresses using \texttt{ompd\_get\_task\_frame}.
%
It can then use \refdef{\texttt{ompd\_get\_scheduling\_ancestor\_task\_region}}{get-scheduling-ancestor-task-region:def} to find the next task in the
thread's stack, and then call \verb|ompd_get_task_frame| for the parent task.
%
The frames between the scheduling parent task's reenter address and the top
task's exit address are frames in which the thread is executing OpenMP
runtime code.
\begin{notes}
JVD: Is this still accurate given John M-C's
proposed new semantics? I think with the new semantics, the addresses
are always for user frames, not OpenMP runtime frames, so ``between''
means \emph{exclusive} of the frame addresses.
\end{notes}
%
This process can be repeated to allow all frames in the thread's
backtrace that correspond to execution in the OpenMP runtime to be
identified.

Using the generating ancestor operation \refdef{\texttt{ompd\_get\_generating\_ancestor\_task\_region}}{get-generating-ancestor-task-region:def},
the same process can be used to determine the chain logical ancestors.
Note that in this case the logical backtrace may traverse
across the stacks of different threads.

The position within the stack frame where the runtime addresses point
is implementation defined.
